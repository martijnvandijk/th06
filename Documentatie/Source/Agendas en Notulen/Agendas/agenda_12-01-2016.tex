\documentclass[dutch]{hu}
\usepackage{makecell}
\title{TH06 Team 12}

\author{Christiaan}{van den Berg}{1660475}
\author{Aydin}{Biber}{1666849}
\author{Martijn}{van Dijk}{1660713}
\author{Chiel}{Douwes}{1666311}

\teacher{Wouter}{van Ooijen}
\teacher{Joost}{Schalken}
\teacher{Marten}{Wensink}
\teacher{Jan}{Zuurbier}
\date{12 januari 2016}
\def \vergaderingDatum{Projectweek}\subtitle{Agenda Teamvergadering \vergaderingDatum}
\begin{document}
\pagestyle{plain}
\maketitle
\chapter{Agenda Teamvergadering \vergaderingDatum}
\section{Opening}
De vergadering wordt formeel geopend, meestal met een welkomstwoord van de voorzitter.

\section{Vaststelling Agendapunten}
De deelnemers kunnen de conceptagenda wijzigen, bijvoorbeeld door de volgorde van de agendapunten te veranderen en nieuwe punten toe te voegen.

\section{Mededelingen}
Onder andere berichten van verhindering.

\section{Vorige Notulen}
Bespreking van de vorige notulen met eventuele correcties.

\section{Ingenomen stukken}
Bespreking van de ingenomen stukken en besluiten over de afhandeling daarvan.

\section{Bespreekpunten}
\begin{itemize}
\item Voortgang
\item Dagplanning
\end{itemize}

\section{Rondvraag}

\end{document}
