\chapter{Onderzoek}
\section{Inleiding}
In dit hoofdstuk bespreken wij de informatie die verzameld is voor het project, en hoe wij aan deze informatie zijn gekomen.
Tevens bespreken wij de onderzoeken en experimenten die zijn uitgevoerd tijdens de realisatie van het systeem.
\newpage

\section{Verzamelde informatie en bronnen}
\subsection{JSON in C++}
De wasprogramma's worden opgeslagen in JSON formaat.
Om uit deze JSON alleen de relevante informatie te halen en via de websocket naar de webinterface te sturen, moet de JSON aan de C++ kant verwerkt worden.
Omdat dit met C++ alleen niet mogelijk is, wordt er gebruik gemaakt van een externe library zoals RapidJSON \cite{repo}.

Er is voor RapidJSON gekozen vanwege zijn performance. Uit benchmark tests is gebleken dat RapidJSON vele malen sneller is dan andere populaire JSON parsers.
Op een systeem wat gebruik maakt van de GCC 32-bit compiler zijn 1000 tests uitgevoerd waarvan de snelheid in milliseconden is opgenomen.
Voor bijvoorbeeld het parsen van DOM had RapidJSON 796 milliseconden nodig, terwijl YAJL 6316 milliseconden nodig had en JsonCpp 6705 milliseconden \cite{performance}.

Voor het toepassen van RapidJSON hebben wij geruik gemaakt van een online tutorial \cite{tutorial}.

\section{Experimenten en resultaten}
\subsection{RapidJSON}
Zoals eerder vermeld maken wij gebruik van RapidJSON vanwege de hoge snelheid en compactheid van de library.
Voor RapidJSON zijn een aantal benchmarks uitgevoerd, maar omdat wij deze niet zelf hebben uitgevoerd zijn de resultaten vernoemd onder de kop "Verzamelde informatie en bronnen".
