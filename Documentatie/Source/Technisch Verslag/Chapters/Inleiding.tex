\chapter{Inleiding}
\section{Opening}
Het bedrijf Swirl\textregistered heeft aangegeven te willen investeren in wasmachine's die op afstand kunnen worden aangezet.
Dit zou gaan door middel van een website die verbonden is met de wasmachine van een klant. Via deze website kan de klant een wasprogramma starten en verschillende parameters van het wasprogramma aanpassen.

\section{Doel}
In dit verslag worden de technische specificaties van het systeem beschreven.
Programmeur's en technici die werken voor Swirl\textregistered kunnen dit document gebruiken om het systeem zo gewenst aan te passen of problemen te verhelpen.

\section{Vooruitblik}
Eerst worden de onderzoeken die zijn uitgevoerd besproken.
Hier geven wij aan welke zaken wij informatie over hebben verzameld en hoe dit is gebeurd. Tevens presenteren wij hier onze experimenten en de resultaten hiervan.

Na het onderzoek bespreken wij het Requirements Architecture. In dit onderdeel worden de belangrijkste aspecten van het Requirements Architecture besproken, evenals de keuzes die gemaakt zijn.

In het Solution Architecture bespreken wij de architectuur van het systeem. Dit wordt gedaan door een aantal modellen die in de loop van het project ontwikkeld zijn. Deze zijn het klassendiagram, concurrency model en het dynamic model.
Tevens leveren wij motivaties voor de gebruikte communicatie protocollen en de taakstructurering.

Na het Solution Architecture wordt de realisatie besproken. Hier beschrijven wij de problemen die tijdens de ontwikkeling zijn opgetreden, en oplossingen voor deze problemen.
Hiernaast geven wij een gedetailleerde uitleg over de algoritmen in de code voor de onderdelen waar dit niet uit de code of commentaar duidelijk is.

Bij de evaluatie bespreken wij zaken die beter ontwikkeld konden worden en wat een mogelijke oplossing hiervoor zou zijn.

Tot slot maken we conclusies en geven hierbij aanbevelingen, gevolgd door bronvermeldingen.