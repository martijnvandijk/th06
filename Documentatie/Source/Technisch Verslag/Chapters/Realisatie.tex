\chapter{Realisatie}
\section{Inleiding}
\newpage

\section{Opgetreden problemen en oplossingen}
\subsection{Schrijven naar bestand d.m.v. JavaScript}
Om de gegevens van de gebruiker op te slaan (en op te halen) moesten wij deze ergens bewaren, bij voorkeur in een bestand.
Tijdens het ontwikkelen van het systeem wat de gebruiker zijn gegevens laat op slaan, kwamen wij er achter dat JavaScript niet in staat is om bestanden aan de server-kant aan te passen.
Dit was een groot probleem omdat wij geen gebruik maken van JavaScript libraries, welke mogelijk wel in staat zijn om bestanden aan te passen.

Uiteindelijk hebben wij dit probleem 2 maal weten te oplossen.
De eerste oplossing was om uiteindelijk toch gebruik te maken van een JavaScript library, in dit geval JQuery.
JQuery kan door middel van een Ajax call (in de vorm van $.ajax) externe en interne scripts aan roepen.
Hiervoor hebben wij een kort PHP script geschreven wat het bestand aan past aan de hand van een binnen komende POST variabele.

De tweede oplossing was, alhoewel simpeler van aard, moeilijker toe te passen, en maakte de eerste verouderd.
Tijdens de ontwikkeling van het systeem kwamen we tot de conclusie dat de bestanden met gebruikers gegevens aan de kant van de WebSocket zouden worden geplaatst.
Dit houdt in dat we door middel van c++ het bestand kunnen aanspreken, waardoor we geen externe library meer hoeven te gebruiken. Tevens hoeven we geen gebruik meer te maken van een PHP script.

\section{Onopgeloste problemen}
Onopgeloste problemen

\section{Gedetailleerde uitleg code}
int i = 0; houdt in dat een integer genaamd "i" wordt geinstantieerd met een waarde van 0.

\section{Uitgevoerde tests en resultaten}
( ☉д⊙)