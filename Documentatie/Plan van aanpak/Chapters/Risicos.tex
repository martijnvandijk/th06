\chapter{Risico's}
Er zijn verschillende risico's die het project in gevaar kunnen brengen. Hier worden die risico's genoemd.
Zie de Kwaliteitsbewaking voor wat de gepaste reactie is op deze risico's.

\section{Uival van de hardware}
Het kan gebeuren dat de hardware niet goed werkt, of compleet uitvalt. Dit vormt een risico voor het project omdat het team dan niet goed de code kan testen. In het geval van falende hardware zal de teamleider zorgen voor vervangende hardware.

\section{Te hoge eisen}
In het geval dat de eisen aan het project te hoog zijn, en niet haalbaar binnen de looptijd van het project zal er door de teamleider contact opgenomen worden met de opdrachtgever om te overleggen welke punten geschrapt of aangepast kunnen worden om de totale werklast te verminderen.

\section{Hardware niet krachtig genoeg}
In het geval dat de hardware van de Raspberry Pi niet krachtig genoeg is om de beschreven functionaliteit te implementeren zal er door de teamleider overlegd worden met de opdrachtgever over het schrappen of aanpassen van functionaliteit, of het ter beschikking stellen van krachtigere hardware. 