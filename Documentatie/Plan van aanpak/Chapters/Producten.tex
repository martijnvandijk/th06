\chapter{Producten}
% \subtitle{Producten}
\section{Inleiding}
Hier worden alle op te leveren producten benoemd die van toepassing zijn tot het project.
\begin{itemize}
	\item Projectdocumentatie
		\begin{itemize}
			\item Teamcontract
			\item Plan van Aanpak
			\item Requirements Document
			\item Requirements Architechture
			\item Solution Architechture
			\item Technisch Verslag
		\end{itemize}
	\item HTML, CSS en JavaScript code van de webpagina's
	\item C++-code van de wasmachine
	\item Demonstratie
\end{itemize}
\section{Projectdocumentatie}
\subsection{Teamcontract}
De afspraken die onderling in het team gemaakt zijn, uiteenlopend van afspraken over beschikbaarheid

\subsection{Plan van Aanpak}
Dit document. Het bevat de basisopzet en -planning van het project.

\subsection{Requirements Document}
Hierin worden de functionele en niet-functionele eisen vermeld die op basis van een interview met de klant zijn vastgesteld, met prioriteiten volgens de MoSCoW methode.

\subsection{Requirements Architecture}
De requirements architecture bevat de functionele systeemeisen: Wat moet de software van het systeem doen? Dit wordt vastgelegd met een of meerdere use case diagrammen. Bij iedere usec case hoort een activity diagram om deze toe te lichten. Verder bevat de Requirements Architecture een Constraints Model. Hierin worden de niet-functionele eisen vastgelegd. 

\subsection{Solution Architecture}
De solution architecture bevat een klassenmodel, een concurrency model en een dynamisch model. Het klassenmodel beschrijft de soorten objecten in het systeem, en welke attributen methodes de klassen hebben. Verder beschrijft het klassenmodel de relatie tussen de verschillende klassen in het systeem. Het concurrency model beschrijft hoe de objecten in het systeem samenwerken en data uitwisselen. Het dynamische model bevat state transition diagrammen die de werking van controller-klassen beschrijven.

\subsection{Technisch Verslag}
Het technisch verslag wordt in \href{https://cursussen.sharepoint.hu.nl/fnt/35/TCTI-V2THO6-14/Studiemateriaal/Inhoud%20Technisch%20Verslag%20themaopdracht%20Domotica.pdf}{dit document op SharePoint} beschreven. Het informatie over de volgende onderwerpen:
\begin{itemize}
	\item Onderzoek
	\item Requirements Architecture
	\item Solution Architecture
	\item Realisatie
	\item Evaluatie
	\item Conclusies en aanbevelingen
\end{itemize}

\section{HTML, CSS en JavaScript code}
De front-end code van de webinterface die geladen wordt in de webbrowser. Deze code wordt opgeleverd via GitHub.

\section{C++-code van de wasmachine}
De back-end code die de verantwoordelijk is voor het uitvoeren van wasprogramma's. Verder wordt vanuit deze software gecommuniceerd met de webbrowser voor het starten van wasprogramma's en om de status van een draaiend wasprogramma te bekijken. Deze code wordt opgeleverd via GitHub

\section{Demonstratie}
Een live demonstratie van het eindproduct aan de opdrachtgevers. 