\chapter{Producten}
% \subtitle{Producten}
\section{Inleiding}
Hier worden alle op te leveren producten behandeld die van toepassing zijn tot het project.
\begin{itemize}
	\item Gebruikershandleiding wasmachine
	\item Gebruikershandleiding webinterface
	\item Teamcontract
	\item Github Repository
	\item Interview Opdrachtgever
	\item Plan van Aanpak
	\item Requirements Document
	\item Requirements Architechture
	\item Solution Architechture
	\item Technisch Verslag
	\item Eindproduct (software + hardware)
\end{itemize}

% \subtitle{MoSCoW Prioritisering}
\section{Inleiding}
In dit hoofdstuk worden alle op te leveren functionaliteiten behandeld die van toepassing zijn tot het project.
Deze producten variëren van software en hardware tot documentatie en gemaakte notulen.
Van deze producten wordt een prioriteiten lijst gemaakt en een lijst aan kwaliteitseisen waaraan deze documenten moeten voldoen.
Tevens wordt besproken hoe de kwaliteit van deze producten zal worden bewaakt.

\subtitle{Kwaliteitseisen op te leveren producten}
Om er voor te zorgen dat de op te leveren producten goed leesbaar zijn worden er een aantal eisen gesteld aan deze producten.

\begin{itemize}
	\item Documentatie mag geen 3 spelfouten bevatten op een enkele pagina. In dien dit wel het geval is wordt het document niet opgeleverd verklaard.
	\item De webinterface moet voldoen aan de door de MoSCoW beschreven eisen.
\end{itemize}
